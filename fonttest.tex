\documentclass[DIV=10,BCOR=1cm,a4paper,parskip,headsepline,english,11pt]{scrbook}
% \usepackage{libertine}

\usepackage{amsmath, amsthm, amssymb,graphicx}
\usepackage[]{mathspec}
\usepackage{xunicode}
% \usepackage{fontspec}
\usepackage{xltxtra}

% \defaultfontfeatures{Scale=MatchLowercase}
\setromanfont[Mapping=tex-text]{Linux Libertine}
% \setsansfont[Mapping=tex-text]{Myriad Pro}
\setsansfont[Mapping=tex-text]{Linux Biolinum}
\setmonofont[Mapping=tex-text]{DejaVu Sans Mono}
\setmainfont{Linux Libertine}
\setmathfont(Digits,Greek,Latin){Linux Libertine}
\setmathrm{Linux Libertine}

\usepackage[english]{babel}
\begin{document}
\frontmatter

\tableofcontents
\mainmatter
\chapter{Introduction}
A digital x-ray imaging device consists of a source that emits x-ray photons and a detector that measures incident x-ray energy. Distinct materials in the object attenuate x-rays differently. In medical applications for example, the probability to absorb x-rays is higher for dense material such as bones and lower for soft tissue.  One of the most common types of digital x-ray detectors, and the one we consider in this work, uses a scintillator to convert x-rays into visible light that can be detected by a regular digital photodetector. Due to the intermediate conversion of x-ray photons into light, these detectors are called indirect detectors. 


\begin{center}
 1234567890 b \textbf{bt} $\mathbf t$ t \textit{b i}
\end{center}

\begin{align}
 \alpha \beta \gamma \Lambda_0 \mathbf \Lambda_0 Λ_0 \boldsymbol Λ_0  \mathbf Λ_0 a b c d e f \; 123456567890 \sum_{i=0}^{\inf} \left( \begin{bmatrix}
                                                                     1 & 2 \\ \alpha & \mathbf b
                                                                    \end{bmatrix} \right)\\
\mathbb R
\end{align}




\end{document}